\chapter{Conclusiones y trabajo futuro.}

En este capítulo vamos a explicar las conclusiones a las que hemos llegado al final del proyecto y el trabajo que podríamos realizar en un futuro.

\section{Conclusiones.}

Al final del proyecto hemos cumplido con los objetivos que nos marcamos en el momento de la realización del anteproyecto, además haber añadido nuevas características al proyecto, como la creación de nuevos códigos QR generados por medio de la aplicación web para que sean firmadas por el usuario usando la aplicación móvil. También se ha implementado un sistema de verificación de firmas por teceros, para que cualquier persona que lo necesite pueda verificar que la firma realizada por un usuario es válida. Para ser usado en caso de conflicto por alguna de las partes.

Además de los objetivos marcados en el anteproyecto también hemos aprendido otras cosas que aunque no estaban explicitamente indicadas en el anteproyecto, van incluidas intrínsecamente a la hora de realizar un proyecto, como son aprender nuevas tecnologías, arquitecturas, APIS, etc. Con el proyecto hemos aprendido a usar varias APIS como son la que proporcionan para la programación de terminales Android, con lo que ello conlleva, aprender la arquitectura y las posibilidades que ofrece un terminar Android. También hemos tenido que investigar que plataforma como servicio se nos amoldaba mejor a la estructura del proyecto, en la que elegimos Google App Engine y descartamos otras como Windows Azure o Amazon Elastic Compute Cloud (Amazon EC2).

También hemos aprendido lo difícil que es realizar una buena ingeniería del software a un proyecto grande, la dificultad que tiene el medir los plazos y el tiempo que se va a tardar en realizar una tarea, etc.

A parte de las tecnologías principales nombradas en la memoria también hemos tenido que usar otras, que aunque los conocimientos necesarios son muy básicos también hemos tenido que aprender, como puede ser JavaScript para cambios en la web sin recargar la web, CSS para la maquetación de las aplicaciones web, HTML, etc.

Personalmente, el proyecto me ha ayudado a aprender nuevas tecnologías como Android o Google App Engine en profundidad y a afianzar los conocimientos que ya tenía en lenguajes de programación como Java o SQL, en el uso de Eclipse y a la hora de instalar API o SDK, que al principio siempre cuesta un poco. Ha sido una experiencia enriquecedora de cara al futuro, ya que actualmente hay un amplio mercado que necesita programadores y diseñadores de software con conocimientos en Android y Java y cada vez es mercado tiende más al desarrollo de aplicaciones web dejando las versiones de escritorio de lado. Además he aprendido bastante sobre certificados digitales, firmas digitales y criptografía, que en un principio es el tema que más me llamaba la atención del proyecto.

\section{Trabajo futuro.}

Este proyecto es una primera inteción de solucionar un problema de la universidad, pero para nada podría ser viable en caso de que se quiera llevar a la realidad, si se quisiera llevar la primera característica que habría que controlar sería toda la seguridad relacionada con los accesos a las aplicaciones web, las firmas, las gestión en profundidad de los certificados, etc. Pero esto es un tema que daría practicamente para un nuevo proyecto. Lo principal que tendríamos que hacer sería investigar como portar los servidores a algunos en los que la UMA pudiera tener control absoluto en ellos, por lo que quizás habría que reescribir las dos aplicaciones web. Mucho código creo que se podría reutilizar, por ejemplo todo el relacionado con los servlet realizados. Un código que no se podría reutilizar sería el relacionado con las insercciones y consultas a la base de datos que habría que cambiarlo y adaptarlo a la base de datos utilizada.

Si se quiere hacer bien se debería de montar una estructura PKI propia de la UMA para todo el tema de los certificados de clave pública para realizar la firma o buscar alguna solución que asocie un certificado a un usuario y que no haya forma de falsificarlo.

Si se quiere llegar a una mayor cantidad de usuarios se debería de realizar otra aplicación móvil desarrollada para iOS o Windows Phone.

Buscar alguna forma de logueo para que no se necesite una cuenta propiedad de Google, ya sea por medio de logueo mediante federación, enlazandolo con DUMA, con el campus virtual o alguna otra plataforma relacionada con la UMA. Se estuvo mirando lo del logueo mediante federación pero al final se descartó, pero se puede volver sobre este tema en caso de que el proyecto se fuese a llevar a la realidad. De hecho hay forma de conseguir una cuenta de google para usar el correo de la UMA, el calendario, etc, quizás con este método se pueda solucionar el problema del login, pero en el desarrollo del proyecto no se ha mirado nada relacionado.






























